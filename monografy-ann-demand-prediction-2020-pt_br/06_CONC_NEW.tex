\chapter{Conclusão} \label{cap:conclusoes}

    Primeiramente, neste trabalho foi possível avaliar importância da metodologia de divisão do conjunto de dados em séries temporais.Visto que com um conjunto de dados de sazonalidade temporal, é notório que a ordenação dos dados na separação dos conjuntos de treino, teste e validação devem seguir uma ordem cronológica para os modelos aprenderem com o passado e realizarem predições para o futuro. O comportamento anômalo da predição identificada no capitulo de divisão do conjunto de dados da fase 1, demonstrado na figura \ref{fig:pandas_wrong_indexing}, e sua eliminação após a organização correta da série temporal, conforme a figura \ref{fig:pandas_correct_indexing}, reforça a a importancia da sequencia temporal dos dados para o correto aprendizado dos modelos testados.
    
    Com relação ao método de produção de refeições com margem de erro e análise da semana anterior foi possível observar que mesmo com a produção 30\% acima do consumo na semana anterior, no fim de cada semestre, o restaurante do ICT-Unifesp  mais do que 30\% são descartados. Este comportamento oscilatório do consumo e o acréscimo de \textit{outliers} acaba ampliando o erro dos modelos em predizer as refeições. No ano de 2019, seguindo este método, 23 mil refeições foram descartas. Neste trabalho O Modelo RNN\_EXO\_3, que apresentou o maior descarte entre todos os 12 modelos testados, obteve um valor máximo de 8914 descartes.Isso evidencia a necessidade de se implementar métodos eficientes para a produção e planejamento de refeições no restaurante universitário da Unifesp.
    
    Em relação aos ajustes empíricos da topologia dos modelos durante a etapa de validação dos primeiros modelos desenvolvidos foi possível notar a redução do RMSE ao longo do aprofundamento da rede Perceptron para treino e avaliação sob o conjunto de validação, validando a hipótese de que os modelos tem capacidade de aprendizado do problema em relação ao ajuste da topologia dos mesmos.

    Apesar do conjunto de dados conter 2 características que informam a distância em dias para o próximo registro e o registro anterior para os modelos identificarem feriados e recessos prolongados, alguns eventos no calendário, como paralisações, não são muito bem representados, indicando a necessidade de uma exploração mais aprofundada destas características que possam representar melhor este comportamento.

    Avaliando o modelo de melhor desempenho na primeira fase, com validação restrita ao primeiro semestre de 2018, os modelos endógenos se saíram melhor do que os modelos mistos. Isso pode significar que os atributos exógenos foram ruidosos durante o aprendizado aprendizado. Estes atributos são a maioria compostos de sazonalidade anual tais como as climáticas, limitadas às estações do ano. 
    O modelo RNN\_EXO\_1 da 2a fase, obteve o melhor desempenho entre todos os modelos avaliados neste trabalho, porém algumas melhorias são indicadas:
        \begin{itemize}
            \item Aumentar o conjunto de dados para o modelo se ajustar às sazonalidades semestrais e à troca de semestres. Os atributos categóricos que indicam os semestres, dia da semana, bem como os que quantificam recessos (distância registro anterior e posterior) têm potencial de agregar aprendizado nessa questão. Ainda é  necessário uma diversificação maior do conjunto de dados, dado que este modelo foi treinado apenas com 1 período de sazonalidade anual (1 ano para treino, outro ano para validação e um terceiro ano para teste).
            \item Acrescentar atributos de eventos importantes para identificar eventos e paralisações.
            \item Um atributos de cardápio tem potencial de aumentar a qualidade da predição.
            \item Um atributo representando o número de alunos matriculados em cada período de cada dia da semana tem grande potencial de aumentar a predição.
            \item Pesquisas podem ser feitas para uma melhor transformação dos dados de entrada no modelo perceptron, pois são dados discretos, enquanto os dados que entram na camada GRU são temporais (com intervalo de 5 dias).
        \end{itemize}
        
    \section{Conclusões gerais}
        
        O fenômeno mais evidente neste trabalho foi a melhoria significativa em todos os modelos de redes neurais artificiais avaliados, apenas alterando-se a organização do conjunto de dados entre a primeira e segunda fase, sem interferência em qualquer parâmetro ou hiper-parâmetro destes modelos. Assim, denota-se a necessidade de mais estudos e experimentos relacionados com organização de conjunto de dados para predições de séries temporais de consumo. As análises diversas de previsão de demanda para o tema abordado requerem extensos métodos de implementação e estruturação de dados.
        
        A aplicação de métodos de treino com retro-propagação, visto a diversidade de parâmetros no aprendizado de máquina dentro de apenas uma análise, onde pode-se montar infinitas topologias diferentes com base na estrutura dos dados coletados, tem potencial para aumentar a performance dos modelos na previsão das demandas par ao RU. 
        
        As heurísticas sobre a definição de topologia apesar de diversas, não são determinísticas, e o processo requer análise exploratória, subjetiva e empírica sobre o tema ou problema a ser abordado. Todavia, foi notória a eficiência dos modelos de aprendizado de máquina em trabalhos relacionados à restaurantes universitários. Como no caso do RU do ICT-Unifesp não há qualquer modelo atual de previsão e a falta de um modelo causa desperdício de alimentos e prejuízo ao restaurante, a abordagem desta pesquisa e sua continuação com novos métodos torna-se viável e importante para auxiliar em uma gestão assertiva dos recurso.