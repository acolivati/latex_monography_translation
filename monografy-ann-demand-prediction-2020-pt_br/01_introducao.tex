\chapter{Introdução}

% \section{Contextualização e Motivação}

\noindent

Dentre as definições mais aceitas sobre segurança alimentar está a versão cunhada durante a Cúpula Mundial da Alimentação de 1996 \cite{shaw2007world}, que a enuncia como uma situação em que todas as pessoas, a todo momento, tenha condições física, social e econômica de acesso a alimentação segura, saudável e nutritiva para uma vida saudável e ativa. Porém, Segundo a Organização Mundial da Saúde \cite{world2009global} mais de 1 bilhão de pessoas no mundo possuem uma dieta nutritivamente insuficiente e mais que o dobro de pessoas tem carência de micronutrientes.

Segundo \citeonline{webb2006} a segurança alimentar pode ser decomposta em 3 pilares essenciais, disponibilidade, possibilidade de acesso e utilização racional. Estes conceitos são intrinsecamente hierárquicos uma vez que a disponibilidade não garante acesso, que por sua vez não garante sua utilização racional. Com os avanços na produção agrícola e a abertura dos mercados econômicos mundiais, ainda que não tenha solucionado o problema, possibilitou passos largos nos dois primeiros pilares da segurança alimentar supracitados. Assim, nas primeiras décadas do século XXI o terceiro pilar, da utilização racional dos recursos, tem se mostrado um importante fator a ser considerado neste novo mundo globalizado \cite{barrett2010}.

No final do ano de 2019 uma doença altamente transmissível foi identificada na província de Wuhan na china, comumente chamada de o novo Coronavírus (COVID-19), a qual foi declarada estar em estado de  pandemia em março de 2020, e até o presente momento infectou mais de 40 milhões de pessoas e ocasionou 1 milhão de mortes\footnote{\url{https://www.worldometers.info/coronavirus/}}. Ainda que com esforços globais para contenção e remediação da pandemia de COVID-19, não existem vacinas ou curas cientificamente comprovadas e a necessidade de \textit{lockdown} ainda é pertinente para evitar novas ondas de contaminação. Os efeitos da pandemia foram devastadores em praticamente todas as áreas econômicas, principalmente o setor de alimentação que é essencial para a manutenção da vida e não tem espaço para \textit{lockdown} \cite{galanakis2020food}.

No dia 22 de julho de 2020, a Associação Brasileira da Indústria e Consumo de alimentos, realizou uma conferência sobre os impactos da pandemia do COVID-19 na cadeia produtiva brasileira de alimentos \footnote{\url{{https://www.abia.org.br/noticias/abia-participa-de-live-para-discutir-os-impactos-da-pandemia-na-cadeia-produtiva-de-alimentos}}}. Nesta conferência foram reunidos especialistas e executivos do setor agroindustrial para analisar mudanças de comportamento de consumo nos setores terciários que lidam diretamente com o fornecimento de alimentos ao consumidor final. A partir desta análise da nova rotina do consumidor, estão sendo buscados novos métodos de se prever o formato atual de consumo e a nova demanda para estes setores. Por fim, foram discutidos também métodos de produção e abastecimento do agronegócio para o setor terciário.

Assim, neste período caótico, em que a mão de obra global reduziu aproximadamente 25\% durante os primeiros meses de pandemia \cite{huff2015resilient}, vem atona a importância do desenvolvimento de processos que utilizem os recursos alimentícios da forma mais racional possível. Neste contexto, uma área que chama a atenção é a de previsão de demandas indústrias ou de estabelecimentos que lidam com alimentos perecíveis em grande quantidade, uma vez que o alimento tem prazo de validade e a produção por demanda pode evitar armazenamento por longos períodos ou descartes indesejado de alimentos prontos para consumo.

Neste sentido, tornam-se viáveis a previsão de demandas para refeições em restaurantes industriais ou de grande porte, como os restaurantes universitários (RU) em que são ofertadas refeições a preços acessíveis para a comunidade estudantil da instituição, através de subsídios no valor repassado aos alunos. Geralmente os RU's são terceirizados por meio de licitações em que o governo do estado uma porcentagem de cada refeição produzida para a empresa gestora do restaurante. Devido a regulamentações sanitárias, refeições preparadas mas não consumidas até o final do expediente devem ser descartadas para evitar contaminações e garantir a segurança alimentar do consumidor \footnote{\url{https://super.abril.com.br/mundo-estranho/o-que-acontece-com-a-comida-que-sobra-dos-restaurantes/}}. Dessa forma, a produção em excedente (não consumida) de refeições por parte dos RU's gera não somente um gasto exagerado de recursos financeiros públicos mas também uma utilização pouco racional dos recursos alimentícios.

Neste contexto, o restaurante universitário da Unifesp em São José dos Campos é um bom estudo de caso uma vez que vende aproximadamente 90 mil refeições anualmente, em que cada refeição tem o valor fixo de R\$2,50 para alunos e um subsidio médio de aproximadamente R\$9,00 por refeição na ultima década (2011-2019). Com isto, foram subsidiados mais de 6 milhões de reais durante os anos de 2011 a 2019.

Uma forma de se reduzir esse desperdício está na desenvolvimento de métodos de predição de consumo \cite{Lopes2008,Rocha2011}. Normalmente, as abordagens para previsão da demanda de consumo de um restaurante universitário se resumem a análise exploratória de dados coletados, como por exemplo, as vendas computadas na semana ou mês anterior. Além disso, informações externa, denominadas dados exógenos, também podem ser considerados no processo de predição, como por exemplo dados climáticos, dados do calendário anual, feriados, dentre outras informações que podem ser relevantes para a estimativa de consumo.

A predição de consumo de redes restaurantes universitários já foi abordada em outros trabalhos. Por exemplo, no trabalho de \citeonline{Lopes2008} realizado na UFV, é utilizado um modelo de rede neural perceptron de 1 camada oculta e 1 neurônio de saída, utilizando os 5 dias anteriores como parâmetros de predição com erro de 3\% sobre o total consumido, e no trabalho de \citeonline{Rocha2011} realizado na UNESP é utilizado também um modelo de rede neural com 1 camada oculta e 1 neurônio de saída, utilizando apenas 1 parâmetro que informa o número de refeições do dia anterior, e outros parâmetros informando médias de dias anteriores e informações relativas à data de consumo, este modelo obteve erro de 9,5\% calculado sobre o total consumido.

Assim, este trabalho consiste na aplicação de modelos de redes neurais para a previsão de demanda das refeições fornecidas no restaurante universitário da Unifesp no campus de São José dos Campos. Para isso, será utilizado um conjunto de dados históricos disponíveis no sistema da Unifesp e outras informações externas (exógenas) que podem impactar o comportamento de consumo e seu processo de predição.

Os objetivos gerais deste trabalho compõem a construção e comparação de modelos de Redes Neurais Artificiais para a previsão da demanda de refeições do restaurante universitário do ICT-UNIFESP. Especificamente, tem-se como objetivos:
\begin{enumerate}[label=\alph*)]
\item Obter e preprocessar os dados de consumo e venda de refeições do ambiente do restaurante e dados do ambiente externo ao consumo, como por exemplo dados climáticos, visando seu uso como entrada de modelos de aprendizado de máquina para a obtenção das predições de consumo na saída destes modelos;
\item Realizar análises exploratórias e descritivas de todos os dados;
\item Construir e avaliar modelos preditivos de redes neurais;
\item Realizar análises sobre as métricas de predições dos modelos apontando suas características viáveis para uma predição de consumo.
\end{enumerate}

O restante deste documento está organizado da seguinte forma. O capítulo \ref{cap:teoria} introduz a fundamentação necessária a compreensão deste trabalho; A literatura relacionada ao tema é descrita no capítulo \ref{cap:literatura}; O capítulo \ref{cap:metodos} apresenta a metodologia empregada nesse estudo; Os resultados são sumarizados no capítulo \ref{cap:resultados}; As principais conclusões desse trabalho bem como as indicações de investigações futuras são apresentadas no capítulo. \ref{cap:conclusoes}; Por fim, o capítulo \ref{cap:anexo1} apresenta alguns resultados adicionais gerados durante os experimentos e os códigos implementados.