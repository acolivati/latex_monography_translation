\chapter{Introduction}

% \section{Contextualização e Motivação}

\noindent

Among the most accepted definitions of food security is the version coined during the 1996 World Food Summit \cite{shaw2007world}, which enunciates it as a situation where all people, at all times, have physical, social and economic conditions to access safe, healthy and nutritious food for a healthy and active life. However, according to the World Health Organization \cite{world2009global} over 1 billion people in the world have a nutritionally insufficient diet and more than twice as many people necessitate micronutrients.

According \citeonline{webb2006}food security can be decompose into 3 essential pillars, availability, possibility of access and rational utilization. These concepts are intrinsically
hierarchical since the availability does not guarantee access, which in turn does not guarantee its rational use. With the advances in agricultural production and the opening of world economic markets, although it has not solved the problem, it has made possible strides in the first two pillars of food security above-mentioned. Thus, in the first decades of the 21st century, the third pillar of the rational use of resources, has proved to be an important factor to be considered in this new globalized world. \cite{barrett2010}.

At the end of 2019, a highly communicable disease was identified in China's Wuhan province, commonly called the new Coronavirus (COVID-19), which was declared in a pandemic state in March 2020, and has so far infected more than 40 million people and caused 1 million  deaths.\footnote{\url{https://www.worldometers.info/coronavirus/}}. Despite global efforts to contain and remedy the COVID-19 pandemic, there are no scientifically proven vaccines or cures and the need for \textit{lockdown} is still pertinent to avoid new waves of contamination. The effects of the pandemic have been devastating in virtually all economic areas, especially in the food sector which is essential for maintaining life and has no room for \textit{lockdown} \cite{galanakis2020food}.

On 22 July 2020, the Brazilian Association of Food Industry and Consumption, held a conference on the impacts of the COVID-19 pandemic on the Brazilian food production chain\footnote{\url{{https://www.abia.org.br/noticias/abia-participa-de-live-para-discutir-os-impactos-da-pandemia-na-cadeia-produtiva-de-alimentos}}}. In this conference were brought together experts and executives from the agro-industrial sector to analyze changes in consumption behavior in the tertiary sectors that deal directly with the supply of food to the final consumer. From this analysis of the new consumer routine, new forecasting methods are being looked for the current format of consumption and the new demand for these sectors. Finally, methods of production and supply of agribusiness for the tertiary sector were also discussed.

Thus, in this chaotic period, when the global labor has reduced approximately 25\% during the first months of the pandemic
 \cite{huff2015resilient}, comes up the importance of developing processes that use food resources in the most rational way possible. Neste contexto, uma área que chama a atenção é a de previsão de demandas indústrias or from establishments that deal with perishable food in large quantities, since the food has an expiration date and production on demand can avoid storage for long periods or unwanted disposal of ready-to-eat foods.

In this sense, forecasting demands for meals in industrial or large restaurants, such as university refectory (RU) becomes viable, which meals are offered at affordable prices to the student community of the institution, through subsidies in the amount passed on to students. Generally, (RUs) are outsourced through bids in which the state government provides a percentage of each meal produced for the refectory management company. Due to health regulations, meals prepared but not consumed until the end of the working day should be discarded to avoid contamination and ensure the food safety of the consumer\footnote{\url{https://super.abril.com.br/mundo-estranho/o-que-acontece-com-a-comida-que-sobra-dos-restaurantes/}}. In this way, the excess production (not consumed) of meals by the (RU's) generates not only an exaggerated expenditure of public financial resources but also anunreasonable use of food resources.

In this context, Unifesp's university refectory in São José dos Campos is a good case study since it sells approximately 90 thousand meals annually, in which each meal has a fixed value of R\$2,50 for students and an subsidy of approximately R\$9,00 
per meal in the last decade (2011-2019). With this, more than 6 million reais were subsidized during the years 2011 to 2019.

One way to reduce this waste is in the development of consumption prediction methods \cite{Lopes2008,Rocha2011}.Usually, the approaches for forecasting the consumption demand of a university restaurant are summarized in the exploratory analysis of collected data, for example the sales computed in the previous week or month.  In addition, external information, denominated exogenous data, can be also considered in the prediction process, such as climate data, annual calendar data, holidays, among other information that may be relevant for consumption estimation.

The prediction of consumption of university refectory chains has already been discussed in earlier works.For example, in the work of \citeonline{Lopes2008} held at the UFV, a model of a perceptron neural network with 1 hidden layer and 1 output neuron is used, using the previous 5 days as prediction parameters with 3\% error over the total consumed, and in the work of \citeonline{Rocha2011} performed at UNESP is also used a neural network model with 1 hidden layer and 1 exit neuron, using only 1 parameter that informs the number of meals of the previous day, and other parameters informing averages of previous days and information regarding the date of consumption, this model obtained an error of 9,5\% calculated on the total consumed.

So, this work consists in the application of Neural Network models to forecast the demand for the meals provided at the university refectory of Unifesp in the campus of São José dos Campos. For this, it will be used a set of historical data available in the Unifesp system and other external (exogenous) information that can impact the consumption behavior and its prediction process.

The general objectives of this work include the construction and comparison of Artificial Neural Networks models for forecasting the demand for meals at the ICT-UNIFESP university refectory. Specifically, it has as objectives:
\begin{enumerate}[label=\alph*)]
\item Obtain and preprocess data of consumption and sale of meals from the refectory environment and data from the external environment to consumption, such as climate data, aiming their use as input of machine learning models to obtain the consumption predictions at the output of these models;
\item Execute exploratory and descriptive analysis of all data;
\item Build and evaluate predictive models of Neural Networks;
\item Execute analysis on the metrics of predictions of the models pointing their viable characteristics for a consumption prediction.
\end{enumerate}

The rest of this document is organized as following. The chapter \ref{cap:teoria} introduces the necessary foundation for the understanding of this work; The literature related to the subject is described in chapter \ref{cap:literatura}; The chapter \ref{cap:metodos} introduces the methodology used in this study; The results are summarized in chapter \ref{cap:resultados}; The main conclusions of this project as well as indications of future investigations are presented in chapter \ref{cap:conclusoes}; Lastly, chapter \ref{cap:anexo1} presents some additional results generated during the experiments and the implemented codes.