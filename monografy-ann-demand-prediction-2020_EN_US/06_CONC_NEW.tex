\chapter{Conclusion} \label{cap:conclusoes}

    First, in this monograph it was possible to evaluate the importance of the methodology to divide the data set into time series. Since with a data set of temporal seasonality it becomes clear that the ordering of the data in the separation of training, testing and validation sets must follow a chronological order for the models to learn from the past and make predictions for the future. The anomalous behavior of the prediction identified in the chapter of division of the phase 1 data set, demonstrated in the figure \ref{fig:pandas_wrong_indexing}, and its elimination after the correct organization of the time series, as shown in the figure \ref{fig:pandas_correct_indexing}, emphasizes the importance of the temporal sequence of the data for the correct learning of the tested models.
    
    Regarding the method of production of meals with margin of error and analysis of the previous week it was possible to observe that even with the production 30\% above the consumption in the previous week, at the end of each semester, the ICT-Unifesp restaurant more than 30\% are thrown out. This oscillatory behavior of consumption and the addition of outliers ends up amplifying the error of the models in predicting meals. In 2019, following this method, 23 thousand meals were discarded. In this project, the RNN\_EXO\_3, Model, which presented the largest disposal among all 12 models tested, obtained a maximum value of 8914 disposals. This highlights the need to implement efficient methods for the production and planning of meals at the university refectory of Unifesp.
    
    In relation to the empirical adjustments of the topology of the models during the validation stage of the first developed models, it was possible to notice the reduction of the RMSE along the deepening of the Perceptron network for training and evaluation under the validation set, validating the hypothesis that the models are capable of learning the problem in relation to the adjustment of their topology.

    Although the data set contains 2 characteristics that inform the distance in days to the next record and the previous record for the models to identify holidays and long recesses, some events in the calendar, such as stoppages, are not very well represented, indicating the need for further exploration of these characteristics that can better represent this behavior.

    Reviewing the best performing model in the first phase, with validation restricted to the first half of 2018, the endogenous models did better than the mixed models. This may mean that the exogenous attributes were noisy during learning. These attributes are mostly composed of annual seasonality such as weather, limited to the seasons. 
    The model RNN\_EXO\_1 of the 2nd phase,  had the best performance among all the models evaluated in this work, but some improvements are indicated:
        \begin{itemize}
            \item Expand the data set for the model to adjust to semester seasonalities and the exchange of semesters. Categorical attributes that indicate semesters, day of the week, as well as those that quantify recesses (previous and subsequent recording distance) have the potential to add learning to this issue. A greater diversification of the data set is still needed, since this model was trained with only 1 annual seasonality period (1 year for training, another year for validation and a third year for testing).
            \item Add important event attributes to identify events and shutdowns.
            \item A menu attribute has the potential to increase the quality of the prediction.
            \item An attribute representing the number of students registered in each period of each day of the week has great potential to increase prediction.
            \item Surveys can be done for a better transformation of the input data in the perceptron model, because they are discrete data, while the data entering the GRU layer are temporal (with an interval of 5 days).
        \end{itemize}
        
    \section{General conclusions}
        
        The most evident phenomenon in this study was the significant improvement in all the evaluated artificial neural network models, only changing the organization of the data set between the first and second phase, without interfering in any parameter or hyper parameter of these models. This means that more studies and experiments related to data set organization are needed for predictions of consumption time series. The diverse analysis of demand forecasting for the subject requires extensive methods of implementation and data structuring.
        
        The application of training methods with backward propagation, considering the diversity of parameters in machine learning within just one analysis where infinite different topologies can be assembled based on the structure of the collected data has the potential to increase the performance of the models in predicting the demands for the RU. 
        
        The heuristics on the definition of topology, although diverse, are not deterministic, and the process requires exploratory, subjective and empirical analysis on the theme or problem to be addressed. However, the efficiency of the machine learning models in work related to university restaurants has been noted. As in the case of the ICT-Unifesp RU there is no current forecasting model and the lack of a model causes food waste and restaurant damage, the approach of this research and its continuation with new methods becomes viable and important to support an assertive management of resources.